% --- algorithm
\usepackage{algorithm}
\usepackage{algorithmic}

% --- code
\usepackage{listings}

% --- colour
\usepackage[table]{xcolor} % \textcolor, \definecolor
\definecolor{bilibili}{RGB}{251, 114, 153}
\definecolor{phub}{RGB}{255, 163, 26}
\definecolor{citegreen}{RGB}{34, 139, 34} % FROM: Mask R-CNN (https://arxiv.org/abs/1703.06870)
\definecolor{royalblue}{RGB}{0, 113, 188}
\definecolor{halfblue}{RGB}{0, 0, 128} % ICML citation blue

% --- encoding
\usepackage[T1]{fontenc}
\usepackage[utf8]{inputenc}

% --- figure
\usepackage{graphicx}
%\usepackage{subfigure}
\usepackage{subcaption} % use it to include subfigure
% wrap figure / table
\usepackage{wrapfig}

% --- table
\usepackage{bigstrut}
\usepackage{booktabs}
\usepackage{caption}
\usepackage{extarrows}
\usepackage{makecell}
\usepackage{multirow}
\usepackage{colortbl} % \cellcolor
\usepackage{pgfplotstable} % dynamically colour cells according to their value (with custom colouring command)

% --- math
\usepackage{amsmath,amssymb,amsfonts,amsthm}
% \usepackage{bm} % defines \bm
\usepackage{mathtools} % hollow char
\usepackage{stmaryrd} % hollow bracket

% argmax, argmin
\DeclareMathOperator*{\argmin}{\arg\min}
\DeclareMathOperator*{\argmax}{\arg\max}
% paired delimiter, add `*' to enable auto-sizing, e.g.
%     \paren{x} -> (x)
%     \paren*{x} -> \left( x \right)
\DeclarePairedDelimiter{\iverson}{\llbracket}{\rrbracket} % Iverson bracket
\DeclarePairedDelimiter{\ceil}{\lceil}{\rceil}
\DeclarePairedDelimiter{\floor}{\lfloor}{\rfloor}
\DeclarePairedDelimiter{\paren}{(}{)} % PARENtheses
\DeclarePairedDelimiter{\sqrbrk}{[}{]} % SQuaRe BRacKets
\DeclarePairedDelimiter{\curbrk}{\{}{\}} % CURly BRacKets
\DeclarePairedDelimiter{\card}{|}{|} % CARDinality
\DeclarePairedDelimiter{\norm}{\|}{\|}

% bm
\ifdefined\bm
  \PackageWarning{\jobname}{\string\bm\space is already defined.}
\else
  \newcommand{\bm}{\boldsymbol}
\fi

% --- symbol
\usepackage{pifont} % \ding{51} (✓), \ding{55} (✗)
\usepackage{pmboxdraw} % \textSFii (└), \textSFviii (├), \textSFx (─)

% --- text
\usepackage{soul} % \st: stridethrough text
\usepackage{lipsum} % generate dummy text

% --- table of content
\usepackage{etoc}

% --- misc
\usepackage{xspace}
%\usepackage{xparse} % parsing multiple arguments in custom commands

% --- reference
\usepackage[capitalize]{cleveref}
\Crefname{section}{Section}{Sections}
\Crefname{table}{Table}{Tables}
\Crefname{figure}{Figure}{Figures}

\crefname{section}{Sec.}{Secs.}
\crefname{table}{Tab.}{Tabs.}
\crefname{figure}{Fig.}{Figs.}

% Easy reference of figure, table, section
% - `\eqref`: built in the `amsmath` package
% - `\nameref`: built in the `nameref` package
% - `\pageref`: built-in
\providecommand{\eqnref}[1]{Eq.~\eqref{#1}}
\providecommand{\Eqnref}[1]{Equation~\eqref{#1}}
\providecommand{\figref}[1]{Fig.~\ref{#1}}
\providecommand{\Figref}[1]{\figurename~\ref{#1}}
\providecommand{\tabref}[1]{Tab.~\ref{#1}}
\providecommand{\Tabref}[1]{\tablename~\ref{#1}}
\providecommand{\secref}[1]{\S\ref{#1}}
\providecommand{\Secref}[1]{Section~\ref{#1}}

% Fake \cite & \ref
% In rebuttal, when you want to \cite or \ref something from the paper
% but you can't have real cross-reference cuz it is in another separate file,
% use hand-craft style to mimic them.
\providecommand{\fref}[1]{\textcolor{red}{#1}} % mod reference colour here to match \ref
\providecommand{\feqref}[1]{(\fref{#1})}
\providecommand{\feqnref}[1]{Eq.~\feqref{#1}}
\providecommand{\ffigref}[1]{Fig.~\fref{#1}}
\providecommand{\ftabref}[1]{Tab.~\fref{#1}}
\providecommand{\fsecref}[1]{\S\fref{#1}}
%\providecommand{\fcite}[1]{$\!${\scriptsize [\textcolor{citegreen}{#1}]}}
\makeatletter
\providecommand{\fcite}[1]{[\@fcite#1,\@nil]}
\def\@fcite#1,#2\@nil{%
  \textcolor{citegreen}{#1}% mod citation colour here to match \cite
  \ifx\@nil#2\@nil\else,\@fcite#2\@nil\fi%
}
\makeatother


% --- hyperref (at last?)
\usepackage[
  pagebackref,
  breaklinks,
  colorlinks,
  bookmarks=true,
  citecolor=citegreen % custom colour defined above
]{hyperref}


% abbreviations
\makeatletter
\DeclareRobustCommand\onedot{\futurelet\@let@token\@onedot}
\def\@onedot{\ifx\@let@token.\else.\null\fi\xspace}
\def\aka{\emph{a.k.a}\onedot}
\def\eg{\emph{e.g}\onedot} \def\Eg{\emph{E.g}\onedot}
\def\ie{\emph{i.e}\onedot} \def\Ie{\emph{I.e}\onedot}
\def\viz{\emph{viz}\onedot}
\def\cf{\emph{cf}\onedot} \def\Cf{\emph{Cf}\onedot}
\def\etc{\emph{etc}\onedot}
\def\vs{\emph{vs}\onedot} % `vs.` not `v.s.`
\def\wrt{w.r.t\onedot}
\def\dof{d.o.f\onedot}
\def\iid{i.i.d\onedot}
\def\wolog{w.l.o.g\onedot}
\def\etal{\emph{et al}\onedot}
\makeatother
